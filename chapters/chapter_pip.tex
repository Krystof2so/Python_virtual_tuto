\chapter[\textit{pip}]{\textit{pip} \\ Maîtriser l'art de la gestion \\ des paquets Python}

\insertcitation{La première règle de l'écologie, c'est que les éléments sont tous liés les uns aux autres.}{Barry Commoner}
\bigskip

Dans le vaste écosystème de Python, \textbf{pip}\footnote{Documentation officielle : \url{https://pip.pypa.io/en/stable/}} se révèle comme étant un outil indispensable. En tant que gestionnaire de paquets par défaut, pip permet d'installer, de mettre à jour et de supprimer des bibliothèques Python de manière simple et efficace. Que l'on travaille sur un petit projet personnel ou sur une application complexe, \textbf{pip} offre la flexibilité et la puissance nécessaires pour gérer les dépendances des projets avec précision.

L'un des principaux avantages de \textbf{pip} est sa simplicité d'utilisation. Avec des commandes intuitives, nous pouvons installer des paquets en quelques secondes, explorer de nouvelles bibliothèques. De plus, \textbf{pip} est compatible avec une multitude de dépôts de paquets, offrant ainsi un accès à une vaste gamme de bibliothèques et d'outils développés par la communauté Python.

Dans ce chapitre, j'aborderai les fonctionnalités essentielles de \textbf{pip} et comment tirer le meilleur parti de cet outil pour optimiser le flux de travail de développement. 

\section{Utilisation dans un environnement virtuel}
Pour éviter d'installer des paquets directement dans l'installation Python du système, il est recommandé d'utiliser un environnement virtuel. Tous les paquets utilisés dans cet environnement seront alors indépendants de l'interpréteur du système. 

Cela présente trois avantages principaux :
\begin{itemize}
    \item L'assurance d'utiliser la bonne version de Python pour le projet en cours.
    \item L'assurance de se référer à la bonne instance de \textbf{pip}.
    \item L'utilisation d'une version de paquetage spécifique pour un projet sans affecter les autres projets.
\end{itemize}

\section{Installer des paquets avec \textit{pip}}
Le langage Python dispose d'une bibliothèque standard, mais également de paquets publiés sur le \textbf{Python Package Index}\footnote{\url{https://pypi.org/}}, également connu sous le nom de \textbf{PyPI}, qui  héberge une vaste collection de paquets, y compris des cadres (\textit{framework}) de développement, des outils et des bibliothèques.

\subsection*{Utilisation de \textit{Python Package Index}}
Une fois l'environnement virtuel créé et activé, toutes les commandes \textbf{pip} alors exécutées le seront dans l'environnement virtuel.

Pour installer des paquets, \textbf{pip} fournit la commande \texttt{install} : 
\begin{lstlisting}[style=terminal]
(.venv)|\textcolor{terminalprompt}{user@machine: \$}| python3 -m pip install nom_paquet
\end{lstlisting}

La commande \textbf{pip} recherche le paquet dans \textbf{PyPI}, résout ses dépendances (dépendances répertoriées dans les métadonnées du paquet) et installe tout ce qui est nécessaire dans l'environnement virtuel actuel. C'est toujours la dernière version du paquetage qui est installée. 

Il est également possible d'installer plusieurs paquets en une seule commande :
\begin{lstlisting}[style=terminal]
(.venv)|\textcolor{terminalprompt}{user@machine: \$}| python3 -m pip install nom_paquet1 nom_p
    aquet2
\end{lstlisting}

La commande \texttt{list} permet d'afficher les paquets installés dans l'environnement, ainsi que leurs numéros de version. Exemple :
\begin{lstlisting}[style=terminal]
(.venv)|\textcolor{terminalprompt}{user@machine: \$}| python3 -m pip list
Package           Version
----------------- --------
jedi              0.19.2
packaging         24.2
parso             0.8.4
pip               25.0
pudb              2024.1.3
Pygments          2.19.1
typing_extensions 4.13.1
urwid             2.6.16
urwid_readline    0.15.1
wcwidth           0.2.13
\end{lstlisting}

Consulter les informations (métadonnées d'un paquet) à l'aide de la commande \textit{show}:
\begin{lstlisting}[style=terminal]
(.venv)|\textcolor{terminalprompt}{user@machine: \$}| python3 -m pip show jedi
Name: jedi
Version: 0.19.2
Summary: An autocompletion tool for Python that can be used for text editors.
Home-page: https://github.com/davidhalter/jedi
Author: David Halter
Author-email: davidhalter88@gmail.com
License: MIT
Location: /home/user/projet/.virtual_py/lib/python3.13/site-packages
Requires: parso
Required-by: pudb
\end{lstlisting}

\subsection*{Utilisation d’un \textit{package index} personnalisé}
Par défaut, \textbf{pip} utilise \textbf{PyPI} pour rechercher des paquets, mais il est également possible de définir un index personnalisé (disponible par exemple sur un serveur privé). Un \textit{package index} personnalisé doit être conforme avec la norme définie dans la \texttt{PEP 503}\footnote{API de dépôt simple : \url{https://peps.python.org/pep-0503/}} pour fonctionner avec \textbf{pip}. Tout index personnalisé qui suit la même API peut être ciblé avec l’option \texttt{--index-url} ou \texttt{-i}.

Par exemple, pour installer l’outil \textbf{rptree} à partir de l’index des paquets \texttt{TestPyPI}\footnote{\url{https://test.pypi.org/}} :
\begin{lstlisting}[style=terminal]
(.venv)|\textcolor{terminalprompt}{user@machine: \$}| python3 -m pip install -i https://test.pypi.org/simple/ rptree
Looking in indexes: https://test.pypi.org/simple/
Collecting rptree
  Using cached https://test-files.pythonhosted.org/packages/0d/bd/83a44ae145bd05b4b667aca5be7b5a7cf61d0d26577c0c93f2a2d0d c2c59/rptree-0.1.0-py3-none-any.whl.metadata (1.5 kB)
Using cached https://test-files.pythonhosted.org/packages/0d/bd/83a44ae145bd05b4b667aca5be7b5a7cf61d0d26577c0c93f2a2d0dc2 c59/rptree-0.1.0-py3-none-any.whl (4.4 kB)
Installing collected packages: rptree
Successfully installed rptree-0.1.0
\end{lstlisting}

\subsection*{Installation de paquets depuis des dépôts \textit{Git}}
\textbf{pip} fournit également l’option d’installer des paquets à partir d’un dépôt \textbf{Git}. 
\begin{lstlisting}[style=terminal]
(.venv)|\textcolor{terminalprompt}{user@machine: \$}| python3 -m pip install git+https://github.com/realpython/rptree
Collecting git+https://github.com/realpython/rptree
  Cloning https://github.com/realpython/rptree to /tmp/pip-req-build-ugz_ee9b
  Running command git clone --filter=blob:none --quiet https://github.com/realpython/rptree /tmp/pip-req-build-ugz_ee9b
  Resolved https://github.com/realpython/rptree to commit 08dcb019aea6e3e326c39be9741e52adb77a2c19
  Installing build dependencies ... done
  Getting requirements to build wheel ... done
  Preparing metadata (pyproject.toml) ... done
Building wheels for collected packages: rptree
  Building wheel for rptree (pyproject.toml) ... done
  Created wheel for rptree: filename=rptree-0.1.1-py3-none-any.whl size=5098 sha256=ea7c230770b418e6c83cc3ab48525c934051303713ea1fd195110 9e1c41a4451
  Stored in directory: /tmp/pip-ephem-wheel-cache-td03c1n1/wheels/bd/b3/32/3714ad3461eca52f1805ec11248c17c60e77c1ae8746f79179
Successfully built rptree
Installing collected packages: rptree
Successfully installed rptree-0.1.1
\end{lstlisting}

Le schéma \texttt{git+https} pointe vers le dépôt \texttt{Git} qui contient un paquet installable.

Pour faire appel à d'autres dépôts \textbf{VCS} voir la page de documentation dédiée\footnote{\url{https://pip.pypa.io/en/stable/topics/vcs-support/}}.

\section{Le fichier d'exigence}
Les dépendances externes sont aussi appelées exigences. Souvent, les projets Python épinglent leurs exigences dans un fichier appelé \texttt{requirements.txt} ou similaire. Le format de fichier \textit{requirements} permet de spécifier précisément quels paquets et versions doivent être installés.

\textbf{pip} dispose d'une commande de gel (\texttt{freeze}) qui affiche les paquets installés au format requis, et qui permet également de rediriger la sortie vers un fichier afin de générer un fichier d’exigences :
\begin{lstlisting}[style=terminal]
(.venv)|\textcolor{terminalprompt}{user@machine: \$}| python3 -m pip freeze > requirements.txt
\end{lstlisting}

Exemple de fichier \texttt{requirements.txt} que l'on peut obtenir après avoir installé les paquets \textbf{pyside6} et \textbf{requests}:
\begin{verbatim}
    certifi==2025.4.26
    charset-normalizer==3.4.2
    idna==3.10
    PySide6==6.9.0
    PySide6_Addons==6.9.0
    PySide6_Essentials==6.9.0
    requests==2.32.3
    shiboken6==6.9.0
    urllib3==2.4.0
\end{verbatim}

Ce fichier peut éventuellement porter un autre nom, cependant une convention largement adoptée consiste à lui donner le nom \texttt{requirements.txt}.

Nous pouvons demander à \textbf{pip} d’installer les paquets listés dans \texttt{requirements.txt} dans un nouvel environnement:
\begin{lstlisting}[style=terminal]
|\textcolor{terminalprompt}{user@machine: \$}| python3 -m venv .venv/
|\textcolor{terminalprompt}{user@machine: \$}| source .venv/bin/activate
|(.venv)\textcolor{terminalprompt}{user@machine: \$}| python3 -m pip install -r requirements
    .txt
|\textcolor{terminalprompt}{user@machine: \$}| python3 -m pip list
Collecting certifi==2025.4.26 (from -r requirements.txt (line 1))
  Using cached certifi-2025.4.26-py3-none-any.whl.metadata (2.5 kB)
Collecting charset-normalizer==3.4.2 (from -r requirements.txt (line 2))
  Using cached charset_normalizer-3.4.2-cp313-cp313-manylinux_2_17_x86_64.manylinux2014_x86_64.whl.metadata (35 kB)
Collecting idna==3.10 (from -r requirements.txt (line 3))
  Using cached idna-3.10-py3-none-any.whl.metadata (10 kB)
Collecting PySide6==6.9.0 (from -r requirements.txt (line 4))
  Using cached PySide6-6.9.0-cp39-abi3-manylinux_2_28_x86_64.whl.metadata (5.5 kB)
Collecting PySide6_Addons==6.9.0 (from -r requirements.txt (line 5))
  Using cached PySide6_Addons-6.9.0-cp39-abi3-manylinux_2_28_x86_64.whl.metadata (4.2 kB)
Collecting PySide6_Essentials==6.9.0 (from -r requirements.txt (line 6))
  Using cached PySide6_Essentials-6.9.0-cp39-abi3-manylinux_2_28_x86_64.whl.metadata (3.9 kB)
Collecting requests==2.32.3 (from -r requirements.txt (line 7))
  Using cached requests-2.32.3-py3-none-any.whl.metadata (4.6 kB)
Collecting shiboken6==6.9.0 (from -r requirements.txt (line 8))
  Using cached shiboken6-6.9.0-cp39-abi3-manylinux_2_28_x86_64.whl.metadata (2.7 kB)
Collecting urllib3==2.4.0 (from -r requirements.txt (line 9))
  Using cached urllib3-2.4.0-py3-none-any.whl.metadata (6.5 kB)
Using cached certifi-2025.4.26-py3-none-any.whl (159 kB)
Using cached charset_normalizer-3.4.2-cp313-cp313-manylinux_2_17_x86_64.manylinux2014_x86_64.whl (148 kB)
Using cached idna-3.10-py3-none-any.whl (70 kB)
Using cached PySide6-6.9.0-cp39-abi3-manylinux_2_28_x86_64.whl (558 kB)
Using cached PySide6_Addons-6.9.0-cp39-abi3-manylinux_2_28_x86_64.whl (166.3 MB)
Using cached PySide6_Essentials-6.9.0-cp39-abi3-manylinux_2_28_x86_64.whl (94.2 MB)
Using cached requests-2.32.3-py3-none-any.whl (64 kB)
Using cached urllib3-2.4.0-py3-none-any.whl (128 kB)
Using cached shiboken6-6.9.0-cp39-abi3-manylinux_2_28_x86_64.whl (206 kB)
Installing collected packages: urllib3, shiboken6, idna, charset-normalizer, certifi, requests, PySide6_Essentials, PySide6_Addons, PySide6
Successfully installed PySide6-6.9.0 PySide6_Addons-6.9.0 PySide6_Essentials-6.9.0 certifi-2025.4.26 charset-normalizer-3.4.2 idna-3.10 requests-2.32.3 shiboken6-6.9.0 urllib3-2.4.0
\end{lstlisting}

Nous pouvons désormais soumettre \texttt{requirements.txt} dans un système de contrôle de version comme \textbf{Git} et l’utiliser pour répliquer le même environnement sur d’autres machines.

\subsection*{Formats du fichier d'exigence}
Chaque ligne du fichier d'exigence indique quelque chose à installer, ou des arguments pour \texttt{pip install}.

Il est possible de spécifier des exigences sous forme de noms simples :
\begin{verbatim}
    certifi
    charset-normalizer
\end{verbatim}

Il est toutefois possible de spécifier les versions de dépendances à l’aide d’opérateurs de comparaison qui donnent une certaine flexibilité pour assurer la mise à jour des paquets tout en définissant la version de base d’un paquet.

Pour cela, éditer \texttt{requirements.txt} afin de transformer les opérateurs d’égalité (\textit{==}) en plus grand ou égal  (\textit{>=}) :
\begin{verbatim}
    certifi==2025.4.26
    charset-normalizer==3.4.2
    idna==3.10
    PySide6>=6.9.0
    PySide6_Addons>=6.9.0
    # ...etc...
\end{verbatim}

A noter le symbole \texttt{\#} qui permet d'insérer des commentaires.

Ensuite, nous pouvons mettre à niveau les packages en exécutant la commande \texttt{install} avec le commutateur \texttt{--upgrade} ou le raccourci \texttt{-U} :
\begin{lstlisting}[style=terminal]
|(.venv)\textcolor{terminalprompt}{user@machine: \$}| python3 -m pip install -U -r requiremen
    ts.txt
\end{lstlisting}

Si une nouvelle version est disponible pour un paquet répertorié, le paquet sera mis à niveau.

Nous pouvons également modifier le fichier des exigences pour empêcher l’installation d'une version donnée ou ultérieure :
\begin{verbatim}
    PySide6>=6.9.0, <6.9.5
    PySide6_Addons>=6.9.0, <6.9.5
\end{verbatim}

Page de la documentation officielle permettant de visualiser divers exemples de formats de fichier d'exigence : \url{https://pip.pypa.io/en/stable/cli/pip_install/#pip-install-examples}.

\subsection*{Séparation des dépendances de production et de développement}
Pour cela on va créer un second fichier d'exigences, \texttt{requirementsi\_dev.txt}, pour lister les outils supplémentaires (par exemple \textbf{pytest} permettant de mettre en place un environnement de développement :
\begin{verbatim}
pytest>=x.y.z
\end{verbatim}

Avoir deux fichiers de configuration exigera d'utiliser \textbf{pip} pour installer les deux: \texttt{requirements.txt} et \texttt{requirements\_dev.txt}. Cependant, \textbf{pip} permet de spécifier des paramètres supplémentaires dans un fichier d'exigences, ainsi, il est possible de modifier \texttt{requirements\_dev.txt} pour installer également les exigences à partir du fichier production \texttt{requirements.txt} :
\begin{verbatim}
-r requirements.txt
pytest>=x.y.z
\end{verbatim}

Maintenant, dans l'environnement de développement, il suffit d’exécuter cette seule commande pour installer toutes les exigences :
\begin{lstlisting}[style=terminal]
|(.venv)\textcolor{terminalprompt}{user@machine: \$}| python3 -m pip install -r requirements
    _dev.txt
\end{lstlisting}

Dans un environnement de production, il suffit uniquement d’installer les exigences de production, soit d'utiliser le fichier \texttt{requirements.txt}.

\subsection*{Geler les exigences pour la production}
Une fois le projet prêt pour la publication, il nous suffira de créer le fichier \texttt{requirement} \texttt{s\_lock.txt}.

\section{Désinstaller des paquets}
Lorsque que l'on installe un paquet il arrive que des dépendances (d'autres paquets) soient installées. Plus on installe de paquets, plus il y a de chances que plusieurs paquets aient des dépendances communes. C’est là que la commande \texttt{show} s’avère utile. Ainsi, avant de désinstaller un paquet, on s'assure d’exécuter la commande \texttt{show} pour ce paquet. Si le champ \texttt{Required-by} est vide, cela signifie qu'il n'y a aucune interdépendance avec un autre paquet. Par contre il faut également lancer la commande \texttt{show} sur les autres dépendances pour procéder à la même vérification. Une fois vérifié l’ordre des dépendances des paquets que nous souhaitons désinstaller, nous pouvez les supprimer en utilisant la commande \texttt{uninstall} :
\begin{lstlisting}[style=terminal]
|(.venv)\textcolor{terminalprompt}{user@machine: \$}| python3 -m pip uninstall paquet
\end{lstlisting}

La commande de désinstallation montre les fichiers qui seront supprimés et demande confirmation. Il est possible de passer outre cette demande de confirmation :
\begin{lstlisting}[style=terminal]
|(.venv)\textcolor{terminalprompt}{user@machine: \$}| python3 -m pip uninstall paquet -y
\end{lstlisting}

Supprimer plusieurs paquets en même temps en passant toujours outre la demande de confirmation :
\begin{lstlisting}[style=terminal]
|(.venv)\textcolor{terminalprompt}{user@machine: \$}| python3 -m pip uninstall -y paquet1 paq
    uet2 paquet3
\end{lstlisting}

Il est également possible de désinstaller tous les packages répertoriés dans le fichier des exigences en fournissant l’option \texttt{-r requirements.txt} :
\begin{lstlisting}[style=terminal]
|(.venv)\textcolor{terminalprompt}{user@machine: \$}| python3 -m pip uninstall -r requirements.
    txt -y 
\end{lstlisting}

Remarque : en travaillant dans un environnement virtuel, il peut être moins compliqué de supprimer l'environnement virtuel et d’en créer un nouveau. Ensuite, nous installerons les paquets dont nous avons besoin au lieu d’essayer de désinstaller les paquets dont nous n’avons pas besoin.

Utiliser \texttt{pip uninstall} est un bon moyen de désinstaller un paquet  du système si celui-ci a été installé accidentellement.
\bigskip

\begin{center}
    \pgfornament[width=0.3\textwidth]{88} % 88 est le numéro de l'ornement
\end{center}

Nous venons d'explorer les multiples facettes de \textbf{pip}, en découvrant comment installer, mettre à jour et gérer les dépendances de nos projets avec facilité et précision. Cependant, la gestion des environnements de développement ne s'arrête pas là. Pour aller plus loin et optimiser encore davantage le flux de travail, il est essentiel de maîtriser les outils qui permettent de gérer plusieurs versions de Python sur un même système. C'est là que \textbf{pyenv} entre en jeu. 

