\chapter[\textit{pyenv}]{\textit{pyenv} \\ Simplifier la gestion \\ des versions de Python}

\insertcitation{Sois comme l’eau : elle s’adapte à toute forme sans jamais perdre sa nature.}{Citation taoïste}
\bigskip

À mesure que l’on progresse dans la pratique du langage Python, une réalité s’impose : tous les projets ne parlent pas le même dialecte. Certains réclament une version ancienne, d’autres tirent parti des nouveautés les plus récentes. Installer plusieurs versions de Python sur une même machine peut alors devenir source de confusion, voire de conflit.

C’est ici qu’intervient \textbf{pyenv}, un outil discret mais redoutablement efficace, qui permet de jongler aisément entre les versions de Python. À la manière de l’eau qui épouse la forme du vase sans jamais perdre sa nature, \textbf{pyenv} s’adapte à chaque projet, chaque environnement, sans rien imposer au système global.

Ce chapitre nous guidera pas à pas dans la découverte et l’utilisation de \textbf{pyenv} : de son installation à sa maîtrise au quotidien. L’objectif n’est pas seulement de fournir un outil de plus, mais de proposer une nouvelle manière d’interagir avec notre environnement de développement — plus souple, plus propre, et infiniment plus adaptée.


