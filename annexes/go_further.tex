\chapter{Pour aller plus loin}

\section*{Les environnements virtuels}
\begin{coloredenum}
    \item PEP 405 - Environnements virtuels Python :\\ \url{https://peps.python.org/pep-0405/}
    \item \textbf{venv} - Création d'environnements virtuels :\\ \url{https://docs.python.org/fr/3.13/library/venv.html}
    \item Environnements virtuels Python - Un abécédaire : \\ \url{https://realpython.com/python-virtual-environments-a-primer/}
\end{coloredenum}

\section*{\textit{virtualenv}}
\begin{coloredenum}
    \item Documentation officielle :\\ \url{https://virtualenv.pypa.io/en/latest/index.html}
    \item Il n'y a pas de magie : édition \textbf{virtualenv} :\\ \url{https://www.recurse.com/blog/14-there-is-no-magic-virtualenv-edition}
\end{coloredenum}

\section*{\textit{pip}}
\begin{coloredenum}
    \item Documentation officielle :\\ \url{https://pip.pypa.io/en/stable/}
    \item Utiliser \textbf{pip} de Python pour gérer les dépendances de vos projets :\\ \url{https://realpython.com/what-is-pip/}
\end{coloredenum}

\section*{\textit{pyenv}}
\begin{coloredenum}
\item Page \textbf{GitHub} du projet \textbf{pyenv} :\\ \url{https://github.com/pyenv/pyenv}
    \item Gérer plusieurs versions de Python avec \textbf{pyenv} :\\ \url{https://realpython.com/intro-to-pyenv/}
\end{coloredenum}
