\chapter[L'alliance \textit{pyenv} / \textit{poetry}]{L'alliance \textit{pyenv} / \textit{poetry}\\ Pour un développement Python \\ propre et efficace}

Voyons comment tirer pleinement parti de l’alliance entre \textbf{pyenv} et \textbf{poetry} sur un système \textbf{GNU/Linux}, en particulier sous \textbf{Debian}, afin de mettre en place un environnement de développement Python propre, isolé et maintenable.

\section*{Installation des outils}
\subsection*{Prérequis}
Le système doit être dôté d'un certains nombres de paquets nécessaires à l'utilisation de ces deux outils :
\begin{lstlisting}[style=bash]
|\userprompt| sudo apt update && sudo apt install -y curl git make build-essential libssl-dev zlib1g-dev libbz2-dev libreadline-dev libsqlite3-dev wget llvm libffi-dev libncursesw5-dev xz-utils libxml2-dev libxmlsec1-dev liblzma-dev tk-dev
\end{lstlisting}

\subsection*{Installation de \textit{pyenv}}
Cloner le dépôt \textbf{pyenv} dans le dossier personnel de l'utilisateur :
\begin{lstlisting}[style=bash]
|\userprompt| curl https://pyenv.run | bash
\end{lstlisting}

Ce script va installer :
\begin{itemize}
    \item \textbf{pyenv}
    \item \textbf{pyenv-doctor}
    \item \textbf{pyenv-virtualenv}
    \item \textbf{pyenv-update}
\end{itemize}
\medskip

Pour utiliser \textbf{pyenv} avec \textbf{zsh} :
\begin{lstlisting}[style=bash]
|\userprompt| echo 'export PYENV_ROOT="$HOME/.pyenv"' >> ~/.zshrc
|\userprompt| echo '[[ -d $PYENV_ROOT/bin ]] && export PATH="$PYENV_ROOT/bin:$PATH"' >> ~/.zshrc
|\userprompt| echo 'eval "$(pyenv init - zsh)"' >> ~/.zshrc 
\end{lstlisting}

Recharger le \textit{shell} :
\begin{lstlisting}[style=bash]
|\userprompt| exec "|\$|SHELL"
\end{lstlisting}

\subsection*{Installation de \textit{poetry}}
Lancer l’installation via le script officiel :
\begin{lstlisting}[style=bash]
|\userprompt| curl -sSL https://install.python-poetry.org |\textbar| python3 -

Retrieving Poetry metadata

# Welcome to Poetry!

This will download and install the latest version of Poetry,
a dependency and package manager for Python.

It will add the `poetry` command to Poetry's bin directory, located at:

/home/user/.local/bin

You can uninstall at any time by executing this script with the --uninstall option,
and these changes will be reverted.

Installing Poetry (2.1.3): Done

Poetry (2.1.3) is installed now. Great!

You can test that everything is set up by executing:

`poetry --version`
\end{lstlisting}

Ajouter \textbf{poetry} au \texttt{PATH} dans le fichier de configuration du \textit{shell} (si ce n'est déjà fait) :
\begin{lstlisting}[style=file]
export PATH="|\$|HOME/.local/bin:|\$|PATH"
\end{lstlisting}

Puis recharger le \textit{shell}.

\subsection*{Vérifier les installations}
\begin{lstlisting}[style=bash]
|\userprompt| pyenv --version
pyenv 2.6.2
|\userprompt| poetry --version
Poetry (version 2.1.3)
\end{lstlisting}

\subsection*{Un outil de mise à jour}
Il sera pour cela intéressant d'ajouter une fonctionnalité de mise à jour automatique de ces outils dans \texttt{.zshrc} (ou tout autre fichier de configuration du \textit{shell} utilisé). Voir par exemple : \url{https://github.com/Krystof2so/dotfiles/blob/main/zsh_functions/devtools.zsh} 

\section*{Flux de travail pour créer notre environnement de développement}
Prenons comme exemple la création d'un environnement de développement avec la dernière version de \textbf{python3.13} disponible.

\subsection*{Installer la dernière version souhaitée}
Vérifions les versions \textbf{3.13} disponibles :
\begin{lstlisting}[style=bash]
|\userprompt| pyenv install --list |\textbar| grep 3.13.
\end{lstlisting}

Voici la liste des versions obtenues :
\begin{verbatim}
3.13.0
3.13.0t
3.13-dev
3.13t-dev
3.13.1
3.13.1t
3.13.2
3.13.2t
3.13.3
3.13.3t
3.13.4
3.13.4t
3.13.5
3.13.5t
pypy2.7-7.3.13-src
pypy3.9-7.3.13-src
pypy3.10-7.3.13-src
\end{verbatim}

Nous disposons donc, comme dernières versions, de \textbf{3.13.5} et \textbf{3.13.5t}. Le \texttt{t} signifie qu'il s'agit de la version de \textbf{python3.13.5} avec le support de \textbf{tkinter}\footnote{\url{https://docs.python.org/fr/3.13/library/tkinter.html}} activé. 

Installons la dernière version de \textbf{python3.13} :
\begin{lstlisting}[style=bash]
|\userprompt| pyenv install 3.13.5  # L'installation prend un certain temps...
\end{lstlisting}

\subsection*{Préparons le répertoire du projet}
\begin{lstlisting}[style=bash]
|\userprompt| poetry new mon_projet --name code_projet
Created package code_projet in mon_projet
\end{lstlisting}

Ce qui nous donne l'architecture suivante :
\begin{lstlisting}[style=tree]
.
|\textbar|-- pyproject.toml
|\textbar|-- README.md
|\textbar|-- src
|\textbar|   |\textbar|-- code_project
|\textbar|       |\textbar|-- __init__.py
|\textbar|   |\textbar|-- tests
        |\textbar|-- __init__.py
\end{lstlisting}

\begin{lstlisting}[style=bash]
|\userprompt| pyenv local 3.13.5  # Spécifier la version de Python
\end{lstlisting}

Le fichier \texttt{.python-version} est alors créé.

\subsection*{Création de l'environnement virtuel}
Pour que notre environnement virtuel soit inséré dans le répertoire du projet :
\begin{lstlisting}[style=bash]
|\userprompt| poetry config virtualenvs.in-project true --local
\end{lstlisting}

\textbf{Poetry} va automatiquement créer un environnement virtuel mais on veut s’assurer qu’il utilise bien la version de Python souhaitée :
\begin{lstlisting}[style=bash]
|\userprompt| poetry env use ~/.pyenv/versions/3.13.5/bin/python
Creating virtualenv code-projet in /home/user/mon_projet/.venv
Using virtualenv: /home/user/mon_projet/.venv
\end{lstlisting}

Voici maintenant le contenu du répertoire du projet :
\begin{verbatim}
poetry.toml
pyproject.toml
.python-version
README.md  
src/  
tests/  
.venv/
\end{verbatim}

Notre environnement de développement est désormais prêt..
