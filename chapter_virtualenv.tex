\chapter[\textit{virtualenv}]{\textit{virtualenv} - Doper ses environnements virtuels}

\textbf{virtualenv}\footnote{\url{https://virtualenv.pypa.io/en/latest/user_guide.html}} est un outil spécialement conçu pour créer des environnements Python isolés. Il s'agit d'un sur-ensemble de \textbf{venv}, offrant des fonctionnalités supplémentaires.

Installation sur \textbf{Debian GNU/Linux} :
\begin{lstlisting}[style=terminal]
|\textcolor{terminalprompt}{user@machine: \#}| aptitude install python3-virtualenv
\end{lstlisting}

Création et activation de l'environnement virtuel (depuis le répertoire du projet):
\begin{lstlisting}[style=terminal]
|\textcolor{terminalprompt}{user@machine: \$}| virtualenv .venv/
created virtual environment CPython3.13.3.final.0-64 in 288ms
  creator CPython3Posix(dest=/home/user/mon_projet/.venv, clear=False, no_vcs_ignore=False, global=False)
  seeder FromAppData(download=False, pip=bundle, via=copy, app_data_dir=/home/user/.local/share/virtualenv)
    added seed packages: pip==25.1.1
  activators BashActivator,CShellActivator,FishActivator,NushellActivator,PowerShellActivator,PythonActivator
|\textcolor{terminalprompt}{user@machine: \$}| source .venv/bin/activate
(venv) |\textcolor{terminalprompt}{user@machine: \$}|
\end{lstlisting}

\textbf{virtualenv} crée l'environnement isolé beaucoup plus rapidement que le module \textbf{venv} intégré, ce qui est possible parce que l'outil met en cache les données d'application spécifiques à la plate-forme, qu'il peut lire rapidement.

\Huge \textit{...A poursuivre...}

